\chapter{Posters}

\section*{P3. Deciphering Head and Neck Cancer Microenvironment: Single-cell and Spatial Transcriptomics Reveals Human Papillomavirus-Associated Differences}

\begin{center}
Hansong Lee\textsuperscript{1}, Sohee Park\textsuperscript{2}, Ju Hyun Yun\textsuperscript{3}, Jeon Yeob Jang\textsuperscript{3}, Yun Hak Kim\textsuperscript{4} \\
\vspace{0.3cm}
\textit{\textsuperscript{1}Medical Research Institute, Pusan National University, Yangsan, Republic of Korea, 50612; \textsuperscript{2}Data Science Center, Insilicogen, Inc., Yongin-si 16954, Korea; \textsuperscript{3}Department of Otolaryngology, School of Medicine, Ajou University, Suwon, Korea, 16499; \textsuperscript{4}Department of Anatomy, School of Medicine, Pusan National University, Yangsan, Republic of Korea, 50612} \\
\vspace{0.3cm}
\textbf{Keywords:} Head and neck squamous cell carcinoma, Human Papillomavirus, Single-cell RNA sequencing, Spatial transcriptomics
\end{center}

\noindent
Human papillomavirus (HPV) is a major causative factor of head and neck squamous cell carcinoma (HNSCC), and the incidence of HPV-associated HNSCC is increasing. The role of tumor microenvironment (TME) in viral infection and metastasis needs to be explored further. We studied the molecular characteristics of primary tumors (PTs) and lymph node metastatic tumors (LNMTs) by stratifying them based on their HPV status. Eight samples for single-cell RNA profiling and six samples for spatial transcriptomics (ST), composed of matched primary tumors (PT) and lymph node metastases (LNMT), were collected from both HPV-negative (HPV−) and HPV-positive (HPV+) patients. Using the 10x Genomics Visium platform, integrative analyses with single-cell RNA sequencing were performed. Intracellular and intercellular alterations were analyzed, and the findings were confirmed using experimental validation and publicly available dataset. The HPV+ tissues were composed of a substantial amount of lymphoid cells regardless of the presence or absence of metastasis, whereas the HPV− tissue exhibited remarkable changes in the number of macrophages and plasma cells, particularly in the LNMT. From both single-cell RNA and ST dataset, we discovered a central gene, pyruvate kinase muscle isoform 1/2 (PKM2), which is closely associated with the stemness of cancer stem cell-like populations in LNMT of HPV− tissue. The consistent expression was observed in HPV− HNSCC cell line and the knockdown of PKM2 weakened spheroid formation ability. Furthermore, we found an ectopic lymphoid structure morphology and clinical effects of the structure in ST slide of the HPV+ patients and verified their presence in tumor tissue using immunohistochemistry. Finally, the ephrin-A (EPHA2) pathway was detected as important signals in angiogenesis for HPV− patients from single-cell RNA and ST profiles, and knockdown of EPHA2 declined the cell migration. Our study described the distinct cellular composition and molecular alterations in primary and metastatic sites in HNSCC patients based on their HPV status. These results provide insights into HNSCC biology in the context of HPV infection and its potential clinical implications.
\newpage

\section*{P5. Enhancing the human interactome through de novo gene network inference from a single-cell atlas}

\begin{center}
Eui Jeong Sung\textsuperscript{1,*}, Jun Ha Cha\textsuperscript{1}, Insuk Lee\textsuperscript{1} \\
\vspace{0.3cm}
\textit{\textsuperscript{1}Department of Biotechnology, College of Life Science and Biotechnology, Yonsei University, Seoul 03722, Republic of Korea; *\textbf{Email:} matlab@yonsei.ac.kr} \\
\vspace{0.3cm}
\textbf{Keywords:} Single-cell RNA sequencing, Functional gene network
\end{center}

\noindent
The investigation of gene-gene relationships using network analysis has been a longstanding practice. However, due to variations in gene expression levels across tissue and cell types, it is essential to construct networks that identify accurate gene associations in tissue and cell type-specific biological contexts. To address this, we used three different types of single-cell RNA sequencing (scRNA-seq) data from multiple studies to construct 100 tissue and cell-type specific networks. We modified scRNA-seq data using three existing imputation and transformation tools respectively and constructed gene networks. Then three individual networks were integrated to reduce technical bias and create a more robust network that distinguishes concealed biological relationships. We integrated our de novo links to the reference network and confirmed that our links reflected context-specific biology leading to improved disease prediction. In summary, we constructed de novo comprehensive networks derived from single-cell atlas, enabling the detection of more context-specific gene correlations. Moreover, we expect our networks to be effective references for comparing networks from other patients and identifying crucial gene modules for therapy.
\newpage

\section*{P7. Single-cell analysis of multiple cancers in the upper gastrointestinal tract reveals characteristics of tumor microenvironments linked to the predictive biomarkers for immunotherapy}

\begin{center}
Seungbyn Baek\textsuperscript{1,*}, Gamin Kim\textsuperscript{2}, Martin Hemberg\textsuperscript{3}, Seong Yong Park\textsuperscript{4,5,**}, Hye Ryun Kim\textsuperscript{2,**}, Insuk Lee\textsuperscript{1,6,**} \\
\vspace{0.3cm}
\textit{\textsuperscript{1}Department of Biotechnology, College of Life Science and Biotechnology, Yonsei University, Seoul 03722, Republic of Korea; \textsuperscript{2}Division of Medical Oncology, Department of Internal Medicine, Yonsei Cancer Center, Yonsei University College of Medicine, Seoul 03722, Republic of Korea; \textsuperscript{3}Gene Lay Institute of Immunology and Inflammation, Harvard Medical School and Brigham and Women’s Hospital, Boston, MA, USA; \textsuperscript{4}Department of Thoracic and Cardiovascular Surgery, Yonsei University College of Medicine, Seoul 03722, Republic of Korea; \textsuperscript{5}Department of Thoracic and Cardiovascular Surgery, Samsung Medical Center, Sungkyunkwan University School of Medicine, Seoul 06351, South Korea; \textsuperscript{6}POSTECH Biotech Center, Pohang University of Science and Technology (POSTECH), Pohang 37673, Republic of Korea; *\textbf{Presenter:} bsb0613@yonsei.ac.kr; **\textbf{Corresponding authors:} insuklee@yonsei.ac.kr} \\
\vspace{0.3cm}
\textbf{Keywords:} Esophageal cancer, Single-cell analysis, Immunotherapy
\end{center}

\noindent
Esophageal cancer is mainly composed of two subtypes – esophageal squamous cell carcinoma (ESCC) and esophageal adenocarcinoma (EAC) – each with distinct risk factors and cancer phenotypes despite arising from the same organ. The recent study by the Cancer Genome Atlas (TCGA) has revealed their distinct genomic characteristics and similarities to other nearby cancers such as head and neck squamous neck carcinoma (HNSCC) and gastric adenocarcinoma (GAC) for ESCC and EAC, respectively. Furthermore, recent clinical trials with anti-PD-1 monotherapy and combination treatment with anti-CTLA-4 reported varying degrees of responses to immunotherapy among those cancers. Here we performed single-cell RNA sequencing of patients with ESCC, EAC, and HNSCC, and collected additional public datasets for comparative analysis of four cancer types in the upper gastrointestinal tract, especially in connection to responses to immunotherapy. In total, we integrated 35 patient samples from 4 different cohorts to comprehensively analyze malignant cells, stromal/endothelial cells, and immune cells. With the integrative analysis, we systematically compared the similarities and differences among those cancer types and expanded understanding of immune mechanisms at single-cell resolution. For malignant cells, we utilized matrix factorization analysis to identify underlying malignant cell programs related to each cancer type. We confirmed clear separation between malignant cells of squamous epithelial cell origins and glandular epithelial cell origins. We further identified the malignant cell programs related to both cancer cell origins and molecular mechanisms of cancer cells. For immune cells, we identified their underlying immune mechanisms that could explain key differences in responses to immunotherapy. With comprehensive analysis of various immune cell-types and their interactions, we identified several T cell populations and related tumor-associated macrophages that could serve as predictive markers of responses to immunotherapy. To validate our findings, we utilized both bulk and single-cell sequencing datasets of various cancer types with treatments of immune checkpoint inhibitors (ICI) and confirmed significance of those immune signatures and cellular compositions in predicting ICI responses.
\newpage

\section*{P12. Pan-cancer single epithelial cell analysis identified specific and common regulatory features of various cancers.}

\begin{center}
Jaewoo Mo\textsuperscript{1,*}, Jiyeoun Park\textsuperscript{1,*}, Junil Kim\textsuperscript{1,@}, Junho Kang\textsuperscript{2,*}, JongEun Park\textsuperscript{2,*} \\
\vspace{0.3cm}
\textit{\textsuperscript{1}School of Systems Biomedical Science, Soongsil University, 369 Sangdo-Ro, Dongjak-Gu, Seoul 06978, Korea; \textsuperscript{2}Graduate School of Medical Science and Engineering, KAIST, 291, Daehak-ro, Yuseong-gu, Daejeon 34141, Korea; *Equal contribution; @Corresponding author (Tel: +82-2-820-0452, E-mail:junilkim@ssu.ac.kr)} \\
\vspace{0.3cm}
\textbf{Keywords:} Pan-cancer, Gene Regulatory Network
\end{center}

\noindent
Certain transcription factors, when losing their normal functions or becoming hyperactivated, can contribute to the development and progression of cancer. While regulatory factors implicated in cancer development have been consistently studied in specific cancer types, they have been relatively uninvestigated in a Pan-Cancer scale. In this study, we integrated single-cell sequencing data from epithelial cells, encompassing 1.5M cells from 24 organs. In embedding space, we computed the Pseudotime from organ-specific regions to common regions and, using TENENT, calculated the TF – Gene Transfer Entropy along the trajectory for 14 primary organs: Bile duct, Bladder, Breast, Colon, Head and Neck, Kidney, Liver, Lung, Ovary, Pancreas, Prostate, Skin, Stomach, and Thyroid, and combined Regulatory Networks. To discern between unique and shared gene regulator, we ranked 292 transcription factors in the Specific and Common sectors by utilizing the mean and standard deviation of the Transfer Entropy scores. We measured the relative expression differences of the transcription factors observed between tumor cells and normal cells.
\newpage

\section*{P13. Identification of Niche-Specific Gene Signatures between Malignant Tumor Microenvironments by Integrating Single Cell and Spatial Transcriptomics Data}

\begin{center}
Jahanzeb Saqib\textsuperscript{1}, Beomsu Park\textsuperscript{1}, Yunjung Jin\textsuperscript{1}, Junseo Seo\textsuperscript{1}, Jaewoo Mo\textsuperscript{1}, Junil Kim\textsuperscript{1,*} \\
\vspace{0.3cm}
\textit{\textsuperscript{1}Soongsil University, School of Systems Biomedical Science, 369 Sangdo-Ro, Dongjak-Gu, Seoul, Korea (06978); *\textbf{Email:} junilkim@ssu.ac.kr} \\
\vspace{0.3cm}
\textbf{Keywords:} single-cell RNA sequencing, spatial transcriptomics, tumor microenvironments, data integration, niche-specific genes, spatial correlation
\end{center}

\noindent
The tumor microenvironment significantly affects the transcriptomic states of tumor cells. Single-cell RNA sequencing (scRNA-seq) helps elucidate the transcriptomes of individual cancer cells and their neighboring cells. However, cell dissociation results in the loss of information on neighboring cells. To address this challenge and comprehensively assess the gene activity in tissue samples, it is imperative to integrate scRNA-seq with spatial transcriptomics. In our previous study on physically interacting cell sequencing (PIC-seq), we demonstrated that gene expression in single cells is affected by neighboring cell information. In the present study, we proposed a strategy to identify niche-specific gene signatures by harmonizing scRNA-seq and spatial transcriptomic data. This approach was applied to the paired or matched scRNA-seq and Visium platform data of five cancer types: breast cancer, gastrointestinal stromal tumor, liver hepatocellular carcinoma, uterine corpus endometrial carcinoma, and ovarian cancer. We observed distinct gene signatures specific to cellular niches and their neighboring counterparts. Intriguingly, these niche-specific genes display considerable dissimilarity to cell type markers and exhibit unique functional attributes independent of the cancer types. Collectively, these results demonstrate the potential of this integrative approach for identifying novel marker genes and their spatial relationships.
\newpage

\section*{P14. Identifying T-cell-subtype-specific Biomarkers for MSI-H Colorectal Cancer}

\begin{center}
Seokho Myung\textsuperscript{1,*}, Man S. Kim\textsuperscript{1,†} \\
\vspace{0.3cm}
\textit{\textsuperscript{1}Translational-Transdisciplinary Research Center, Clinical Research Institute, Kyung Hee University Hospital at Gangdong, College of Medicine, Kyung Hee University, Seoul, Republic of Korea; †co-corresponding author; *\textbf{Email:} tjzh13@khu.ac.kr} \\
\vspace{0.3cm}
\textbf{Keywords:} Colorectal cancer, MSI-H
\end{center}

\noindent
Colorectal cancer (CRC) is a leading cause of cancer-related deaths worldwide. Patients with microsatellite instability-high (MSI-H) tumors have been shown to have a better response to immunotherapy, and previous study show that tumor-infiltrating lymphocytes (TILs) play a crucial role in this response. In this study, we performed single-cell RNA sequencing of MSI-H and MSS colorectal tumors, focusing on the T cell subgroup. Data was obtained from open source gene expression omnibus GSE222300, GSE179784, and GSE188711. The total of 5 MSI-H(microsatellite instability high) patient, and 6 treatment naïve MSS(microsatellite stable) patient data were collected. We identified several genes that were differentially expressed between the two groups. Looking in to the PPI(protein-protein-interaction) five of those genes that were closely related to DNA mismatch repair genes (MLH1, MSH2, MSH6, PMS2). These findings suggest that the immune response in MSI-H and MSS colorectal tumors may be influenced by differences in DNA mismatch repair gene expression, which could have implications for the development of targeted immunotherapies in the future.
\newpage

\section*{P1. A phospho-switch controls RNF43-mediated degradation of Wnt receptors to suppress tumorigenesis}

\begin{center}
Tadasuke Tsukiyama\textsuperscript{1,*}, Bon-Kyoung Koo\textsuperscript{2,*} \\
\vspace{0.3cm}
\textit{\textsuperscript{1}Department of Biochemistry, Faculty of Medicine and Graduate School of Medicine, Hokkaido University, Japan; \textsuperscript{2}Center for Genome Engineering, Institute for Basic Science, South Korea; *\textbf{Email:} tsukit@med.hokudai.ac.jp; koobk@ibs.re.kr} \\
\vspace{0.3cm}
\textbf{Keywords:} Wnt Signalling, intestinal organoids
\end{center}

\noindent
Frequent mutation of the tumour suppressor RNF43 is observed in many cancers, particularly colon malignancies. RNF43, an E3 ubiquitin ligase, negatively regulates Wnt signalling by inducing degradation of the Wnt receptor Frizzled. In this study, we discover that RNF43 activity requires phosphorylation at a triplet of conserved serines. This phospho-regulation of RNF43 is required for zebrafish development and growth of mouse intestinal organoids. Cancer-associated mutations that abrogate RNF43 phosphorylation cooperate with active Ras to promote tumorigenesis by abolishing the inhibitory function of RNF43 in Wnt signaling while maintaining its inhibitory function in p53 signalling. Our data suggest that RNF43 mutations cooperate with KRAS mutations to promote multi-step tumorigenesis via the Wnt-Ras-p53 axis in human colon cancers. Lastly, phosphomimetic substitutions of the serine trio restored the tumour suppressive activity of extracellular oncogenic mutants. Therefore, harnessing phospho-regulation of RNF43 might be a potential therapeutic strategy for tumours with RNF43 mutations.
\newpage

\section*{P2. Single-cell analysis reveals immune dysregulation linked to poor prognosis in Fusobacterium nucleatum-infected colorectal cancer}

\begin{center}
Il Seok Choi\textsuperscript{1,*}, Kyung-A Kim\textsuperscript{2}, Yoon Dae Han\textsuperscript{3}, Sang Cheol Kim\textsuperscript{4}, Han Sang Kim\textsuperscript{2}, Insuk Lee\textsuperscript{1} \\
\vspace{0.3cm}
\textit{\textsuperscript{1}Department of Biotechnology, College of Life Science and Biotechnology, Yonsei University, Seoul 03722, Republic of Korea; \textsuperscript{2}Division of Medical Oncology, Department of Internal Medicine, Yonsei Cancer Center, Yonsei University College of Medicine, Seoul 03722, Republic of Korea; \textsuperscript{3}Department of Colorectal Surgery, Severance Hospital, Yonsei University College of Medicine, Seoul, Republic of Korea; \textsuperscript{4}Division of Healthcare and Artificial Intelligence, Department of Precision Medicine, Korea National Institute of Health, Republic of Korea; *\textbf{Email:} choiismath@yonsei.ac.kr} \\
\vspace{0.3cm}
\textbf{Keywords:} Single-cell RNA sequencing, Colorectal cancer, Fusobacterium nucleatum
\end{center}

\noindent
Fusobacterium nucleatum (Fn) is commonly detected in colorectal cancer (CRC) and worsens patient survival. Fn appears to play a role in colorectal cancer carcinogenesis through suppression of the antitumor immune response. Aiming to uncover the underlying mechanisms, we collected 42 samples of surgically removed colon tissues from patients newly diagnosed with colon cancer. To analyze the bacterial community composition within these tissues, we utilized amplicon sequencing, targeting the V4 variable region of the 16S rRNA. We performed single-cell RNA sequencing (scRNA-seq) analysis of tumor-infiltrating immune cells from Fn-infected [Fn (+)] and Fn-uninfected [Fn (−)] patients. By utilizing gene expression signature derived from scRNA-seq data and bulk transcriptome profiles of the TCGA cohort, we identified immune cell types associated with Fn infection in CRC. Using RNA velocity and cell-cell interaction analysis of single-cell transcriptome data, we unraveled immune cell subtypes modulated by Fn infection. Furthermore, trajectory-based differential expression analysis and single-cell gene network analysis shed light on how the intratumor immune system is disturbed by Fn infection. A novel gene signature had a poor prognostic impact in patients with Fn-infected tumors, compared with Fn-uninfected patients in the TCGA cohort. Overall, this study identified a novel potential Fn-related immune evasion mechanism, beyond suppression of T cell-mediated immune response. Novel gene signatures with a poor prognostic impact can be used for patient stratification and developing targeted strategies in patients with Fn infection.
\newpage

\section*{P4. Single-cell analysis reveals cellular and molecular factors counteracting HPV-positive oropharyngeal cancer immunotherapy outcomes}

\begin{center}
Junha Cha\textsuperscript{1,†}, Dahee Kim\textsuperscript{2,†}, Gamin Kim\textsuperscript{3,†}, Jae-Won Cho\textsuperscript{4}, Euijeong Sung\textsuperscript{1}, Seungbyn Baek\textsuperscript{1}, Min Hee Hong\textsuperscript{3}, Chang Gon Kim\textsuperscript{3}, Nam Suk Sim\textsuperscript{2}, Hyun Jun Hong\textsuperscript{2}, Jung Eun Lee\textsuperscript{3}, Martin Hemberg\textsuperscript{4}, Seyeon Park\textsuperscript{5}, Sun Ock Yoon\textsuperscript{7}, Sang-Jun Ha\textsuperscript{5,*}, Yoon Woo Koh\textsuperscript{2,*}, Hye Ryun Kim\textsuperscript{3,*}, Insuk Lee\textsuperscript{1,6,*} \\
\vspace{0.3cm}
\textit{\textsuperscript{1}Department of Biotechnology, College of Life Science and Biotechnology, Yonsei University, Seoul 03722, Republic of Korea; \textsuperscript{2}Department of Otorhinolaryngology, Yonsei University College of Medicine, Seoul 03722, Republic of Korea; \textsuperscript{3}Division of Medical Oncology, Department of Internal Medicine, Yonsei Cancer Center, Yonsei University College of Medicine, Seoul 03722, Republic of Korea; \textsuperscript{4}Gene Lay Institute of Immunology and Inflammation, Harvard Medical School and Brigham and Women’s Hospital, Boston, MA 02115, USA; \textsuperscript{5}Department of Biochemistry, College of Life Science and Biotechnology, Yonsei University, Seoul 03722, Republic of Korea; \textsuperscript{6}POSTECH Biotech Center, Pohang University of Science and Technology (POSTECH), Pohang, 37673, Republic of Korea; \textsuperscript{7}Department of Pathology, Yonsei University College of Medicine, Seoul, Republic of Korea; †Equal contribution; *\textbf{Email:} insuklee@yonsei.ac.kr} \\
\vspace{0.3cm}
\textbf{Keywords:} Oropharyngeal squamous cell carcinoma, human papillomavirus, immunotherapy, resident memory T cells, CD161, CLEC2D
\end{center}

\noindent
Oropharyngeal squamous cell carcinoma (OPSCC) induced by human papillomavirus (HPV-positive) is associated with better clinical outcomes than HPV-negative OPSCC. However, the clinical benefits of immunotherapy in patients with HPV-positive OPSCC remain unclear.

To identify the cellular and molecular factors that limited the benefits associated with HPV in OPSCC immunotherapy, we performed single-cell RNA (n =20) and T cell receptor sequencing (n = 10) analyses of tumor samples. Primary findings from our single-cell analysis were confirmed through immunofluorescence experiments, and secondary validation analysis were performed via publicly available transcriptomics datasets.

We found significantly higher transcriptional diversity of malignant cells among non-responders to immunotherapy, regardless of HPV infection status. We also observed a significantly larger proportion of CD4+ follicular helper T cells (Tfh) in HPV-positive tumors, potentially due to enhanced Tfh differentiation. Most importantly, CD8+ resident memory T cells (Trm) with elevated KLRB1 (encoding CD161) expression showed association with dampened antitumor activity in HPV-positive OPSCC patients, which may explain their heterogeneous clinical outcomes. Notably, all HPV-positive patients, whose Trm presented elevated KLRB1 levels, showed low expression of CLEC2D (encoding the CD161 ligand) in B cells, which may reduce tertiary lymphoid structure activity. Immunofluorescence of HPV-positive tumors treated with immune checkpoint blockade showed an inverse correlation between the density of CD161+ Trm and changes in tumor size.

We report that CD161+ Trm counteracts clinical benefits associated with HPV in OPSCC immunotherapy. Our findings suggests that targeted inhibition of CD161 in Trm could enhance the efficacy of immunotherapy in HPV-positive oropharyngeal cancers.
\newpage

\section*{P6. Understanding the molecular mechanisms of muscle atrophy: A single-nucleus transcriptome analysis of mice with dexamethasone-induced skeletal muscle atrophy}

\begin{center}
Bum Suk Kim\textsuperscript{1}, Ahyoung Choi\textsuperscript{1}, Yoomi Baek\textsuperscript{1}, No Soo Kim\textsuperscript{2}, Aeyung Kim\textsuperscript{3}, Haeseung Lee\textsuperscript{4}, Hyunjin Shin\textsuperscript{1,*} \\
\vspace{0.3cm}
\textit{\textsuperscript{1}MOGAM Institute for Biomedical Research, Seoul 06730, Republic of Korea; \textsuperscript{2}Korean Medicine Convergence Research Division, Korea Institute of Oriental Medicine, Daejeon 34054, Republic of Korea; \textsuperscript{3}Korean Medicine Application Center, Korea Institute of Oriental Medicine, Daegu 41062, Republic of Korea; \textsuperscript{4}Department of Pharmacy, Pusan National University, Busan 46241, Republic of Korea; *\textbf{Corresponding author:} hyunjin.shin@mogam.re.kr} \\
\vspace{0.3cm}
\textbf{Keywords:} Muscle atrophy, Dexamethasone, snRNA-seq
\end{center}

\noindent
Skeletal muscle atrophy is the loss of muscle mass and function caused by a variety of factors, including aging, disuse, malnutrition, disease, and drugs. Dexamethasone (DEX)-induced muscle atrophy is a valuable model for understanding muscle atrophy because it is a rapid and reliable way to induce muscle wasting in experimental animals. In this study, we generated single-nucleus RNA-seq (snRNA-seq) data of skeletal muscle from mice with DEX-induced muscle atrophy to decipher cell type-specific transcriptional changes associated with this condition. Our data analysis revealed that repeated DEX treatment led to a decreased number of muscle cells and increased amounts of fibroblasts. These cells showed altered gene expression related to circadian rhythm and fibrosis which are directly associated with muscle mass reduction. Also, we identified potential ligand candidates involved in this process, including muscle regeneration. These findings provide new insights into the molecular mechanisms of muscle atrophy and could inform the development of new strategies for preventing and treating this devastating condition.
\newpage

\section*{P8. Single-cell profiling unveils molecular insights and region-specific heterogeneity in the hippocampus during global ischemia}

\begin{center}
Donghee Kwak\textsuperscript{1,†}, Yun Hak Kim\textsuperscript{2,3,*}, Hong Il Yoo\textsuperscript{4,*} \\
\vspace{0.3cm}
\textit{\textsuperscript{1}Department of Convergence Medical Sciences, School of Medicine, Pusan National University, Yangsan, Republic of Korea; \textsuperscript{2}Department of Biomedical Informatics, School of Medicine, Pusan National University, Yangsan, Republic of Korea; \textsuperscript{3}Department of Anatomy, School of Medicine, Pusan National University, Yangsan, Republic of Korea; \textsuperscript{4}Department of Anatomy and Neuroscience, Eulji University School of Medicine, Daejeon, South Korea; †\textbf{Email:} gdh5137392@pusan.ac.kr} \\
\vspace{0.3cm}
\textbf{Keywords:} Single-cell transcriptomics, Brain, Global ischemia
\end{center}

\noindent
Ischemic stroke is a critical medical condition that predominantly impacts the elderly population. Glial cells, including microglia, astrocytes, and oligodendrocytes, comprise the peri-infarct environment within the central nervous system, playing a crucial role in post-stroke immune regulation. However, emerging evidence suggests a dual impact in ischemic stroke. In this study, we conducted single-cell RNA sequencing in the hippocampal CA1 and CA3/DG regions of both the sham group and a rat 4-vessel surgery model, which serves as a global ischemia model. Clustering 43,202 cells in both Sham and ischemic models revealed 7 distinct cell lineages and 18 clusters. Comparative analyses unveiled changes in gene expression and subpopulation composition based on region and disease presence. Our data substantiated an elevation in pro-inflammatory microglial subtypes following a stroke, accompanied by distinct distribution differences in microglial pathways, depending on regional discrepancies. Moreover, we identified oligodendrocytes as a unique cell subtype specific to stroke conditions. Overall, our study elucidated differences in cell roles and compositions during global ischemia, which had not been comprehensively studied before. It also highlighted variations in the roles and compositions of individual cells based on the specific hippocampal region. Through this research, we aim to resolve questions about stroke resistance in various hippocampal regions, which have hitherto been inaccurately understood and controversial. This comprehension has the potential to streamline applications like drug development by providing insights into the mechanisms associated with ischemic stroke.
\newpage

\section*{P9. Multiregional single-cell transcriptome profiling reveals an association between partial EMT and immunosuppressive microenvironment in oral squamous cell carcinoma}

\begin{center}
Seunghoon Kim\textsuperscript{1,†}, Hyun Jung Kee\textsuperscript{2,†}, Dahee Kim\textsuperscript{2,†}, Jinho Jang\textsuperscript{1,†}, Hyoung-oh Jeong\textsuperscript{1}, Nam Suk Sim\textsuperscript{2}, Taejoo Hwang\textsuperscript{1}, David Whee-Young Choi\textsuperscript{1}, Kyoung Jun Lee\textsuperscript{1}, Jaewoong Lee\textsuperscript{1}, Young Min Park\textsuperscript{2,*}, Semin Lee\textsuperscript{1,*}, Yoon Woo Koh\textsuperscript{2,*} \\
\vspace{0.3cm}
\textit{\textsuperscript{1}Department of Biomedical Engineering, College of Information-Bio Convergence Engineering, Ulsan National Institute of Science and Technology (UNIST); UNIST-Gil 50, Eonyang-eup, Ulju-gun, Ulsan 44919, Republic of Korea; \textsuperscript{2}Department of Otorhinolaryngology, College of Medicine, Yonsei University; 50-1 Yonsei-ro, Seodaemun-gu, Seoul 03722, Republic of Korea; †Equal contribution; *\textbf{Corresponding author}} \\
\vspace{0.3cm}
\textbf{Keywords:} Oral squamous cell carcinoma, Field cancerization, Multiregional single-cell RNA sequencing, Partial EMT, Tumor microenvironment, Prognostic signature
\end{center}

\noindent
Background: Oral squamous cell carcinoma (OSCC) is highly heterogeneous and associated with human papillomavirus (HPV) infection or alcohol and tobacco exposure. Especially, patients with HPV-negative OSCC have a worse prognosis compared with HPV-associated OSCC patients. However, the mechanisms driving tumorigenesis and progression of HPV-negative OSCC are poorly understood. We sought to investigate the association between intratumoral heterogeneity (ITH) and tumor microenvironment (TME) in HPV-negative OSCC development and its clinical implications. Methods: We performed single-cell RNA sequencing on 231,442 cells obtained from the tumor core, tumor periphery, adjacent surrounding tissue, and metastatic lymph node samples of 10 patients with advanced HPV-negative OSCC. Multiplex immunofluorescence staining was performed to validate the signatures and subtypes associated with prognosis. Results: Chromosomal aberrations in epithelial cells were observed in adjacent surrounding tissues, providing evidence of field cancerization in OSCC. The expression of epithelial-to-mesenchymal transition (EMT)-associated genes was distinctly upregulated in the tumor cores and peripheries. Interestingly, partial EMT (p-EMT) cells showed significant activation of glycolysis and hypoxia signatures, which may serve as potential biomarkers for clinical outcomes. IL2RA+ regulatory T cells and CXCL1+ tumor-associated macrophages (TAMs) were substantially enriched in tumor cores, peripheries, and surrounding tissues, shaping an immunosuppressive TME. Moreover, p-EMT scores of epithelial cells positively correlated with M2 scores of TAMs, while the proportion of p-EMT at the tumor periphery was negatively associated with that of GZMB+ exhausted CD8+ T cells with cytotoxic potential and TNFRSF9+ mast cells, conferring adverse prognosis. Conclusions: Our study provides insights into understanding the interplay between ITH and TME of advanced HPV-negative OSCC at the single-cell level with the identification of predictive biomarkers of patients’ outcomes.
\newpage

\section*{P10. Single-cell analysis of psoriasis with coexisting systemic lupus erythematosus reveals microenvironmental network of fibroblasts and mature dendritic cells}

\begin{center}
So-Jung Choi\textsuperscript{1,*}, Jongeun Lee\textsuperscript{1,*}, Yun Jung Huh\textsuperscript{2}, Jaecheon Ko\textsuperscript{3}, Hyun Seung Choi\textsuperscript{1}, Seo-Young Choi\textsuperscript{1}, Hyo Jeong Nam\textsuperscript{1}, Hyun Je Kim\textsuperscript{1,4,†}, Jeong Eun Kim\textsuperscript{2,†} \\
\vspace{0.3cm}
\textit{\textsuperscript{1}SNU} \\
\vspace{0.3cm}
\textbf{Keywords:} Single cell rna-seq, Translational research
\end{center}

\noindent
Systemic lupus erythematosus (SLE) and psoriasis are both chronic immune-mediated diseases. SLE is caused by T-helper 1 (Th1), Th2, Th17, and B cell axis activation, while psoriasis is primarily driven by Th17 activation. The coexistence of PS and SLE has been previously reported. However, their shared pathogenesis has not been fully elucidated and there remains difficulty in providing proper treatment for patients with coexisting PS and SLE.We performed single-cell RNA sequencing with skin biopsies from non-lesion, pre-treatment lesion, and post-treatment lesion at 12 weeks of guselkumab in a patient with concomitant PS and SLE. Transcriptomic profiles for 56,000 number of PS with SLE (non-lesional, lesional, and post-treatment lesional) cells were created. There was increased expression of genes related to interferon (IFN) pathway in both non-lesional and lesional skin and it demonstrated type 17 T cell (Tc17) subpopulation expressed highest level of IL-23 receptors. Interestingly, IL-23 inhibition caused marked gene expression shifts, with fibroblast cluster displaying the most significant decrease in the expression of genes linked to IL-23 signaling pathway. Ligand-receptor analysis revealed pro-inflammatory CCL19+ fibroblasts interacted with mature LAMP3(+) dendritic cells (DC). Importantly, this interaction decreased after IL-23 inhibition. We delineated the cell populations and cell-specific gene expression in the skin of patient with coexisting PS and SLE and provide insight into the transcriptomic landscape before and after the treatment with IL-23 inhibitor, guselkumab.
\newpage

\section*{P11. TENET+ a tool for reconstructing gene networks by integrating single cell expression and chromatin accessibility data}

\begin{center}
Hyeonkyu Kim\textsuperscript{1,*}, Hoebin Chung\textsuperscript{1,\#}, Jiyoung Lee\textsuperscript{1}, Hwisoo Choi\textsuperscript{1}, Wooheon Kim\textsuperscript{1}, Junil Kim\textsuperscript{1,@} \\
\vspace{0.3cm}
\textit{\textsuperscript{1}School of Systems Biomedical Science, Soongsil University, 369 Sangdo-Ro, Dongjak-Gu, Seoul 06978, Republic of Korea; *\textbf{Presenter}; \textsuperscript{\#}Equal contribution; \textsuperscript{@}\textbf{Corresponding author; (Tel +82-2-820-0452, E-mail} junilkim@ssu.ac.kr)} \\
\vspace{0.3cm}
\textbf{Keywords:} Single cell multiomics, scRNAseq, scATACseq, Gene regulatory networks, Epigenetic regulator, TENET
\end{center}

\noindent
Reconstruction of gene regulatory networks (GRNs) is a powerful approach to capture a prioritized gene set controlling cellular processes. In our previous study, we developed TENET a GRN reconstructor from single cell RNA sequencing (scRNAseq). TENET has a superior capability to identify key regulators compared with other algorithms. However, accurate inference of gene regulation is still challenging. Here, we suggest an integrative strategy called TENET+ by combining single cell transcriptome and chromatin accessibility data. TENET+ predicts target genes and regions associated with transcription factors (TFs) and links the target regions to their corresponding target gene. As a result, TENET+ can infer regulatory triplets of TF, target gene, and enhancer. By applying TENET+ to a paired scRNAseq and scATACseq dataset of human peripheral blood mononuclear cells, we found critical regulators and their epigenetic regulations for the differentiations of CD4 T cells, CD8 T cells, B cells and monocytes. Interestingly, not only did TENET+ predict several top regulators of each cell type which were not predicted by the motif-based tool SCENIC, but we also found that TENET+ outperformed SCENIC in prioritizing critical regulators by using a cell type associated gene list. Furthermore, utilizing and modeling regulatory triplets, we can infer a comprehensive epigenetic GRN. In sum, TENET+ is a tool predicting epigenetic gene regulatory programs for various types of datasets in an unbiased way, suggesting that novel epigenetic regulations can be identified by TENET+.
\\
\\
\noindent
Github page: https://github.com/hg0426/TENETPLUS.
\newpage