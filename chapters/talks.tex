\chapter[Talks (22nd, January)]{Talks\\(22nd, January)}

% JHP
\section*{Constructing a mouse splicing atlas using single cell long-read sequencing}

\begin{center}
\index[author]{Park, Jihwan}Jihwan Park \\
\vspace{0.2cm}
\textit{School of Life Sciences, Gwangju Institute of Science and Technology (GIST), Gwangju, Republic of Korea} \\
\end{center}

\noindent
Single-cell long-read sequencing allows the analysis of every piece of biological information from the original full-length cDNA library (e.g., single nucleotide variant, structural variant, alternative splicing patterns, fusion transcripts, transcription start site, and polyadenylation site), creating new possibilities for existing single-cell applications. However, amplification artifacts and high cost hinder the use of single-cell long-read sequencing in biomedical applications. We developed Ouro-seq, a cost-effective single-cell long-read sequencing method that efficiently removes amplification artifacts while simultaneously enriching and depleting targets genes with high specificity. Our method enables a comprehensive analysis of cell type-specific transcript isoforms, the detection of mutations at the single cell level, and targeted sequencing for hundreds of genes or cells of interest. Using Ouro-seq we have constructed a single cell splicing atlas of 12 mouse tissues. We identified a number of de novo transcript isoforms and cell type-specific splicing patterns. This is the first single-cell landscape of transcript isoforms across various tissues, providing a valuable resource for the researchers. In addition, our method enables the re-analysis of existing single-cell sequencing libraries that are stored in individual laboratories, increasing the utilization of existing single-cell genomics resources to investigate important biological questions that are difficult to answer using only short-read sequencing technologies.
\newpage

% BJH
\section*{Scalable, multimodal multiplexed single cell sequencing}

\begin{center}
\index[author]{Hwang, Byungjin}Byungjin Hwang \\
\vspace{0.2cm}
\textit{Department of Biomedical Sciences, College of Medicine, Yonsei University} \\
\end{center}

\noindent
With increasing availability of the single-cell sequencing platforms, researchers are making exciting discoveries to transform biomedical research. However, current technologies can assess only handful of cells (1,000-10,000) due to limitation in the microfluidic encapsulation inefficiency. To date, most of the research has been focused on measuring RNA expression, often missing important phenotyping information. Biologists heavily relied on Flow cytometry or Mass cytometry to phenotype cells with limited number of markers. Here, I will introduce how the combinatorial indexing-based methods can be leveraged to achieve ultra-high throughput profiling of single-cell level multimodal expression patterns to topple technical barriers and discuss further how we envision single-cell perturbation study in genome-wide scale for the first time. As the sequencing cost continues to decrease, ultra high-throughput technology will democratize population-scale single cell studies and provide better insights into cellular interactions and heterogeneity in complex biological processes to delve deeper into human disease pathogenesis.
\newpage

% JKK
\section*{Dissecting cellular heterogeneity and plasticity in adipose tissue}

\begin{center}
\index[author]{Kim, Jong Kyoung}Jong Kyoung Kim \\
\vspace{0.2cm}
\textit{Department of Life Sciences, POSTECH, Pohang, Korea} \\
\end{center}

\noindent
Cell-to-cell variability in gene expression exists even in a homogeneous population of cells. Dissecting such cellular heterogeneity within a biological system is a prerequisite for understanding how a biological system is developed, homeostatically regulated, and responds to external perturbations. Single-cell RNA sequencing (scRNA-seq) allows the quantitative and unbiased characterization of cellular heterogeneity by providing genome-wide molecular profiles from tens of thousands of individual cells. Single-cell sequencing is expanding to combine genomic, epigenomic, and transcriptomic features with environmental cues from the same single cell. In this talk, I demonstrate how scRNA-seq can be applied to dissect cellular heterogeneity and plasticity of adipose tissue. 
\newpage

% JEP
\section*{Fibrotic niche-enriched single-cell transcriptome uncovers novel age-associated changes in liver tissue}

\begin{center}
\index[author]{Park, Jong-Eun}Jong-Eun Park \\
\vspace{0.2cm}
\textit{Graduate School of Medical Science and Engineering, Korea Advanced Institute of Science and Technology (KAIST), Daejeon, Republic of Korea} \\
\end{center}

\noindent
Aging is characterized by the gradual loss of the ability to maintain homeostasis, leading to disrupted physiological integrity and increased vulnerability to chronic disease and death. Cellular senescence has been suggested as a key factor in the aging process, with its impact exerted through the paracrine activities known as senescence-associated secretory phenotype (SASP). Single-cell genomics, as a powerful technique to understand tissue biology, has been applied to uncover age-related changes, in searching for the transcriptional identity of senescent cells in various organs. Surprisingly, many single-cell studies have failed to identify age-associated senescent cells with distinct gene expression patterns, which includes the well-characterized p16-expressing senescent cells. In this study, we developed a fibrotic niche enrichment sequencing (FiNi-seq) method, to enrich cell types previously uncaptured in scRNA-seq due to the dissociation bias. We applied this method to the liver of young and old mice, and discovered novel age-associated cell types. The transcriptomic analysis combined with spatial data revealed that these cells represent an age-associated fibrotic niche, which includes dysfunctional endothelial cells, SMOC1+ fibroblasts, and recruited immune cells. Our new single-cell analysis method can be applied to diverse tissue to reveal micro-fibrotic niche and will shed light to the understanding of the early dynamics of fibrosis.
\newpage

% HSK
\section*{Writing and reading genomes: \\ Functional genomics study with CRISPR and single-cell sequencing}

\begin{center}
\index[author]{Kim, Heon Seok}Heon Seok Kim \\
\vspace{0.2cm}
\textit{Department of Life Science, Hanyang University, Seoul, Republic of Korea} \\
\end{center}

\noindent
Large scale genomic studies are cataloguing thousands of genetic mutations. With the sheer number of discovered mutations, determining their phenotype and functional characterization remains an enormous challenge. Conventional CRISPR screening have been used to determine the phenotypic effects of genetic mutations by analyzing altered cellular fitness. However, these methods were not able to determine CRISPR perturbed cells’ state in detail. Single-cell CRISPR screening enabled transcriptome changes induced by CRISPR engineering. However, they had relied on short-read sequencing which are not able to detect full-length transcripts. Herein, we developed novel single-cell technologies to directly introduce mutations into the human genome and determine their transcriptional phenotype with integrated long- and short-read sequencing among individual cells. These enable introduction of more diverse genome engineering (i.e., gene KO, point mutations, gene fusions, etc.) and investigation of their effect in depth (i.e., genetic mutations, transcript isoform usage, etc.) in a single-cell resolution. This will generate rich and valuable dataset about complex phenotypes of various genetic mutations which were not attainable before with previous methods.
\newpage

% LEB (m)
\section*{Spatial profiling of the human developing head with EEL FISH}

\begin{center}
\index[author]{Borm, Lars E.}Lars E. Borm \\
\vspace{0.2cm}
\textit{Laboratory of Computational Biology, VIB Center for AI \& Computational Biology (VIB.AI), Leuven, Belgium} \\
\end{center}

\noindent
Our tissues are composed of hundreds to thousands of different cell types that are intricately arranged in space to give rise to the function of the organ. Recent advances in spatially resolved transcriptomics can reveal these patterns, however, imaging based approaches suffer from a bottleneck in the imaging time resulting in long experiments and restrictions in spatial throughput. To overcome this issue, we have developed EEL FISH, where the RNA molecules form a tissue section are transferred onto a flat surface using electrophoresis, so that less Z-stacking is needed, and centimetre scale tissues can be rapidly processed. We have applied EEL FISH to map the cells of the entire mouse brain. Furthermore, EEL FISH mitigates issues with tissue autofluorescence so that difficult tissues, such as the adult human brain, can be processed. Lastly, we have used this technology to study the development of the human brain in two samples of 5 and 7.5 weeks after conception, where we reveal the patterning that gives rise to the cellular complexity.
\newpage

% FJT
\section*{Generative AI for modeling single-cell state and response}

\begin{center}
\index[author]{Theis, Fabian J.}Fabian J. Theis \\
\vspace{0.2cm}
\textit{Helmholtz Munich - http://comp.bio}
\end{center}

\noindent
Advances in single cell genomics nowadays allow the large scale construction of organ atlases. These can be used to study perturbations such as signaling, drugs or diseases, with large-scale access to state changes on the multi-omic and spatial level. This provides an ideal application area for machine learning methods to understand cellular response. With generative AI revolutionizing many fields of science by allowing researchers to explore uncharted territories, generate novel hypotheses, and simulate complex phenomena, we ask how it has been enabling modeling single cell variation, potentially towards a single cell foundation model.
 
After reviewing deep generative representation learning approaches to identify the gene expression manifold, I will shortly outline some applications on cell atlas building. Then I will discuss interpretable modeling of perturbations on this manifold, in particular effect of drug responses as well as multiscale readouts such as disease state across patients, and how to learn organism-wide cell type predictors. I will finish with extensions towards temporal and spatial observations.
\newpage

\chapter[Talks (23rd, January)]{Talks\\(23rd, January)}

% BL (m)
\section*{Spatial transcriptomics pipelines for brain cell atlases across species}

\begin{center}
\index[author]{Long, Brian}Brian Long \\
\vspace{0.2cm}
\textit{Allen Institute for Brain Science, Seattle, WA, USA} \\
\end{center}

\noindent
The cellular complexity of the brain is reflected in the diverse gene expression profiles among neurons and non-neuronal cells. In the past decade, single-cell transcriptomics has revealed a rich landscape of cell types in the brain and large-scale transcriptomics studies have produced transcriptomic atlases of larger samples and increased complexity. Alongside this development, spatial transcriptomics technologies are maturing to the point where cell type atlases can now include precise spatial location of cell types within brain structures. Two recent demonstrations of this approach include the whole mouse brain atlas from the BRAIN Initiative Cell Census Network and the atlas of cell types across Alzheimer's disease in the human middle temporal gyrus. 

However, in both of these successful examples, the volume of tissue sampled was small compared to the size of the human brain. How can we scale this approach up to larger brains? This talk will discuss strategies for addressing this problem, present the strategy of the Allen Institute for Brain Science efforts within the BRAIN Initiative Cell Atlas Network and discuss our experience with spatial transcriptomics at scale.
\newpage

% YZ
\section*{Deciphering spatially-resolved cells in mouse brain using explainable artificial intelligence}

\begin{center}
\index[author]{Zhang, Yun (Renee)}Yun (Renee) Zhang \\
\vspace{0.2cm}
\textit{J. Craig Venter Institute, La Jolla, California 92037, USA}
\end{center}

\noindent
With the advent of multiplex fluorescence in situ hybridization (FISH) and in situ RNA sequencing technologies, spatial transcriptomics analysis is advancing rapidly, providing spatial location and gene expression information about cells in tissue sections at single cell resolution. Cell type classification of these spatially-resolved cells can be inferred by matching the spatial transcriptomics data to reference atlases derived from single cell RNA-sequencing (scRNA-seq) in which cell types are defined by their gene expression profiles. However, cell type matching of the spatially-resolved cells to the reference scRNA-seq atlas is challenging due to the intrinsic differences in the technology and cell type granularity between the spatial and scRNA-seq data modalities. In this study, we systematically evaluated six computational algorithms for cell type matching across four image-based spatial transcriptomics experimental protocols (MERFISH, smFISH, BaristaSeq, and ExSeq) conducted on the same mouse primary visual cortex (VISp) brain region. We find that the spatial transcriptomics data showed high variability and very different dynamic ranges across experimental protocols. By combining the individual matching results using explainable artificial intelligence (XAI) strategies, we are able to obtain a strongly agreed consensus cell type matching that aligns with the biological expectations. The consensus cell type matching can also be used for the Spot-based Spatial cell-type Analysis by Multidimensional mRNA density estimation (SSAM) for spatially-resolved cells, which effectively identified the distinct CR type located close to the L4 subclass.
\newpage

% DSL
\section*{Epigenomic and chromosomal architectural reconfiguration in developing human frontal cortex and hippocampus}

\begin{center}
\index[author]{Lee, Dong-Sung}Dong-Sung Lee \\
\vspace{0.2cm}
\textit{Department of Life Science, University of Seoul, Seoul, Republic of Korea} \\
\end{center}

\noindent
The human frontal cortex and hippocampus play critical roles in learning and cognition. We investigated the epigenomic and 3D chromatin conformational reorganization during the development of the frontal cortex and hippocampus, using more than 53,000 joint single-nucleus profiles of chromatin conformation and DNA methylation (sn-m3C-seq). The remodeling of DNA methylation predominantly occurs during late-gestational to early-infant development and is temporally separated from chromatin conformation dynamics. Neurons have a unique Domain-Dominant chromatin conformation that is different from the Compartment-Dominant conformation of glial cells and non-brain tissues. We reconstructed the regulatory programs of cell-type differentiation and found putatively causal common variants for schizophrenia strongly overlap with chromatin loop-connected, cell-type-specific regulatory regions. Our data demonstrate that single-cell 3D-regulome is an effective approach for dissecting neuropsychiatric risk loci.
\newpage

% HYC
\section*{Spatial transcriptomics integrated with imaging analysis}

\begin{center}
\index[author]{Choi, Hongyoon}Hongyoon Choi \\
\vspace{0.2cm}
\textit{Department of Nuclear Medicine, Seoul National University Hospital; CTO, Portrai, Inc.} \\
\end{center}

\noindent
Recent advances in spatial biology have enabled multi-omic data with spatial information, which require new evaluation and utilization methods for clinical applications, such as biomarker and drug development. As another aspect, spatial transcriptomics is an ex vivo multiplexed molecular imaging technique that can map the entire gene expression in a tissue. Therefore, spatial transcriptomics is linked to molecular imaging methods that contain spatially matched data. The integration of image analysis with spatial transcriptomics analysis has the potential to offer various applications. In particular, first, spatial transcriptomics can provide new information when combined with traditional imaging analysis at the clinical or translational level. For example, drug distribution is essential for the refinement and advancement of emerging drug delivery, and spatial transcriptomics allows us to look at this in great detail. Furthermore, this technology can be combined with drug tracking to enable traceability in a variety of drugs, including cellular therapeutics. In addition, interpreting spatial transcriptomics in conjunction with histology images allows for a different approach. It is possible to combine the clinically widely used H\&E images with spatial transcriptomics to devise a model that predicts the core tumor microenvironment based on H\&E images alone, or to automatically segment images such as H\&E for spatial transcriptomics analysis and use them to interpret spatial transcriptomics (IAMSAM). In addition, we can apply image interpretation and analysis methods to spatial transcriptomics. For example, if comparative analysis between different samples is needed, image registration, etc. is required, and spatial transcriptomics can be translated into image data and utilized in an image analysis pipeline to track how parts of an organization change (SpatialSPM). This talk will discuss recent advances in analytic and integrative methods for spatial transcriptomics for translational research. The potential of spatial transcriptomics is vast, and as new methods are developed to address its challenges, it has the potential to revolutionize our understanding of disease and help in the development of precision medicine.
\newpage

% CHS
\section*{Fixative eXchange (FX)-seq: single-nucleus transcriptome profiling of clinical FFPE specimens}

\begin{center}
\index[author]{Sohn, Chang Ho}Chang Ho Sohn \\
\vspace{0.2cm}
\textit{Advanced Science Institute, Graduate program of Nano-BioMedical Engineering, Yonsei University, Seoul, Korea; Center for NanoMedicine, Institute for Basic Science, Yonsei University, Seoul, Korea} \\
\end{center}

\noindent
I will briefly introduce Fixative eXchange (FX)-seq, a new technology developed in our lab for FFPE single nucleus RNA seq. Using FX-seq, we demonstrated its utility and versatility by analyzing a total of 320k nuclei from PFA-fixed and FFPE-treated mouse brain and human cancer tissue. Overall, FX-seq offers the potential to overcome the limitations of current tools by analyzing single-nucleus transcriptomes in archived clinical FFPE samples.
\newpage

% HJ
\section*{Coupling single-cell genome \& epigenome through strand-seq data analysis, and beyond}

\begin{center}
\index[author]{Jeong, Hyobin}Hyobin Jeong \\
\vspace{0.2cm}
\textit{Hanyang Institute of Bioscience and Biotechnology, Hanyang University, Seoul, Republic of Korea; Department Systems Biology, College of Life Science and Biotechnology, Yonsei University, Seoul, 03722, South Korea
} \\
\end{center}

\noindent
Somatic structural variants are widespread in cancer, but their impact on disease evolution is understudied due to a lack of methods to directly characterize their functional consequences. We proposed a computational method, scNOVA, which utilizes Strand-seq to perform haplotype-aware integration of structural variant discovery and molecular phenotyping in single cells, by using nucleosome occupancy to infer gene expression as a read-out. Application to leukemias and cell lines identifies local effects of copy-balanced rearrangements on gene deregulation, and consequences of structural variants on aberrant signaling pathways in subclones. We discovered distinct SV subclones with dysregulated Wnt signaling in a chronic lymphocytic leukemia patient. We further uncovered the consequences of subclonal chromothripsis in T-cell acute lymphoblastic leukemia, which revealed c-Myb activation, enrichment of a primitive cell state and informed successful targeting of the subclone in cell culture, using a Notch inhibitor. Not only cancer system, we can apply this approach to study functional effect of somatic SVs in clonal hematopoiesis and aging. Also, we are now extending scNOVA to develop scalable and broadly applicable bioinformatics methods which can link SVs to their functional effects, to enable systematic single-cell multiomic studies of structural variation in heterogeneous cell populations.
\newpage

% HK
\section*{scLENS: Data-driven signal detection for unbiased scRNA-seq data analysis}

\begin{center}
\index[author]{Kim, Hyun}Hyun Kim \\
\vspace{0.2cm}
\textit{Biomedical Mathematics Group, Institute for Basic Science, Daejeon, Republic of Korea} \\
\end{center}

\noindent
Due to the high-dimensional and noisy nature of scRNA-seq data, dimensionality reduction tools have been developed to extract biologically meaningful signals from the data. However, most of these tools require the user to manually determine the dimension of signals, which can potentially introduce user bias into the analysis results. Furthermore, log normalization, commonly used in the preprocessing step, can unintentionally distort the signals in the data. To circumvent these long-standing problems, we developed scLENS, which accurately captures biological signals from scRNA-seq data without user bias and distortion. Specifically, we identified the primary cause of signal distortion resulting from log normalization and effectively addressed it by integrating L2 normalization into the preprocessing step of scLENS. Moreover, we utilized random matrix theory-based noise filtering and the signal robustness test to identify the data-driven threshold for the dimension of signals. We successfully demonstrated that scLENS outperformed 11 widely used dimensionality reduction tools, particularly for scRNA-seq data characterized by high sparsity and variability. This methodology is expected to enable more accurate downstream analysis by automatically differentiating real signals from scRNA-seq data, which simultaneously obtains much information with complex structures.
\newpage

% VK
\section*{Probabilistic models to resolve cell identity and tissue architecture}

\begin{center}
\index[author]{Kleshchevnikov, Vitalii}Vitalii Kleshchevnikov \\
\vspace{0.2cm}
\textit{Wellcome Sanger Institute} \\
\end{center}

\noindent
Cell identity drives cell-cell communication and tissue architecture and is in return regulated by cell-extrinsic cues. Cell identity is determined by the combination of intrinsic developmentally established transcription factor use (TF) and constitutive as well as cell communication-dependent TF activities. Presented work shows two probabilistic models that we developed to advance the understanding of these processes using single-cell and spatial genomic data.
 
Spatial transcriptomic technologies promise to resolve cellular wiring diagrams of tissues in health and disease, but comprehensive mapping of cell types in situ remains a challenge. Here we present cell2location, a Bayesian model that can resolve fine-grained cell types in spatial transcriptomic data and create comprehensive cellular maps of diverse tissues. Cell2location accounts for technical sources of variation and borrows statistical strength across locations, thereby enabling the integration of single cell and spatial transcriptomics with higher sensitivity and resolution than existing tools. We assess cell2location in three different tissues and demonstrate improved mapping of fine-grained cell types. In the mouse brain, we discover fine regional astrocyte subtypes across the thalamus and hypothalamus. In the human lymph node, we spatially map a rare pre-germinal centre B cell population. In the human gut, we resolve fine immune cell populations in lymphoid follicles. Collectively our results present cell2location as a versatile analysis tool for mapping tissue architectures in a comprehensive manner.
 
Cell identity and plasticity is regulated by a combinatorial code mediated by transcription factors and the cell communication environment. Systematically dissecting how the regulatory code robustly defines the vast complexity of cell populations across tissues is a long-standing challenge. Measured using the assay for transposase-accessible chromatin with sequencing (ATAC-seq), DNA accessibility provides a readout of intermediate gene regulation steps at single-cell resolution, with technologies measuring both RNA and ATAC providing the necessary evidence to build mechanistic models of regulation. Existing methods address one or several subproblems of modelling DNA accessibility. For example, the DNA sequence-based deep learning models represent combinatorial interactions and in-vivo TF-DNA recognition preferences. In contrast, GRN models use TF abundance profiles across cells and in-vitro-derived TF-DNA recognition preferences, optionally incorporating ATAC-seq data as a filter. All models learn cell-type specific weights and properties and don’t generalise to new TF abundance states such as new cell types. Therefore, we are missing an end-to-end mechanistic model that represents all steps of the biological process, that generalises to both new DNA sequences and TF abundance combinations and can simultaneously characterise hundreds to thousands of cell states observed in single-cell genomics atlases. Here, we formulated cell2state, a mechanistic end-to-end probabilistic model of TF recruitment to a chromatin locus and downstream TF effect on DNA accessibility. Cell2state is designed to achieve the generalisation of regulatory predictions to unseen cell types. Cell2state A) estimates TF nuclear protein abundance and models B) how TFs recognise DNA, C) how TF sites in DNA lead to TF recruitment to a chromatin locus, D) how the activity of DNA-associated TFs affects chromatin accessibility. To evaluate generalisation, we defined the computational problem and developed a workflow for predicting the scATAC-seq readout for previously unseen chromosomes and cell types. We show that cell2state outperforms the state-of-the-art deep learning models (ChromDragoNN) at explaining DNA accessibility differences across cells. Finally, to look at cell state plasticity, we developed ways to use cell2state to simulate the possible chromatin states given TF abundance of source cell types.
\newpage
